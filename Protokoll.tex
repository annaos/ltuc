%
% File Name     : Protokoll.tex
% Purpose       :
% Creation Date : 26-10-2013
% Last Modified : Tue 29 Oct 2013 01:20:05 PM CET
% Created By    :
%

\documentclass[a4paper,12pt]{article}
\usepackage{amsfonts}
\usepackage{tabularx}
\usepackage[utf8]{inputenc}
\usepackage[ngerman]{babel}
\usepackage{fancyhdr}
\usepackage{amsmath} 
\usepackage{graphicx} 
\usepackage[hidelinks]{hyperref} 

\pagestyle{fancy}

\fancyhead[R]{\today}
\fancyhead[C]{Andrey,Anna,Oliver}
\fancyhead[L]{Algorithmen}

\begin{document}

\begin{center}
\Large
\textbf{Projekt 1}
\end{center}

\newpage
\tableofcontents
\newpage
\setcounter{tocdepth}{2}


\section{Einleitung}

Im Zuge dieser Projektarbeit soll ein Angriff auf ein einfaches Authentifizierungssystem durchgeführt werden.
Der Angriff soll dabei in zwei Phasen ablaufen, eine Offline-Phase (Precomputation) und eine Online-Phase.


\section{Szenario}

Aus der Aufgabenstellung ergibt sich das folgende Szenario:
[Grafik]
Ein Terminalnutzer meldet sich mit seinem Kennwort an einem Server an. Das eingegebene Kennwort wird mit dem SHA1-Algorithmus
gehasht und die letzten 4 Byte dieses Hashes werden an den Server übertragen. Der Server vergleicht diese 4 Byte mit seiner Datenbank
und gewährt Zugriff, falls der richtige Hash gesendet wurde.

\section{Problemstellung}


\section{Umsetzung}

Wie soll Password generiert werden?

Idee/Aufbau
drei Teilung

\subsection{Idee}

\subsection{Designentscheidung}

\subsubsection{Terminal}

Warum UUID?

\subsubsection{Server}

\subsubsection{Angreifer}

Datenstruktur
Speicherbedarf

Laufzeit Rechenaufwand

\begin{tabular} [h] {||c|c|c|c||} 
\hline \rule[-1.5mm]{0pt}{5.5ex} Tabellengröße & Generieren  & Schreiben & Lesen in ms\\ 
\rule[-1.5mm]{0pt}{5.5ex} n & in ms  & in ms & \\ 
\hline
\hline \rule[-1.5mm]{0pt}{5.5ex} 1000 &  & & \\ 
\hline \rule[-1.5mm]{0pt}{5.5ex} 2000 &  & & \\ 
\hline \rule[-1.5mm]{0pt}{5.5ex} 10000 &  & & \\ 
\hline \rule[-1.5mm]{0pt}{5.5ex} 100000 &  & & \\ 
\hline \rule[-1.5mm]{0pt}{5.5ex} 1000000 &  & & \\ 
\hline \rule[-1.5mm]{0pt}{5.5ex} 2000000 &  & & \\ 
\hline
\end{tabular}

[Gnuplot Grafik]

Erklaeung der benutzten Bibliotheken:
Sockets
Serialisierung
Warum Hashtable?

Geburtstagsparadoxon

\section{Quellen}

\nolinkurl{http://en.wikipedia.org/wiki/Universally_unique_identifier#Random_UUID_probability_of_duplicates}

\end{document}
